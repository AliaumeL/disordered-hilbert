\section{Concluding Remarks}
\label{sec:conclusion}

\AP The results presented in this paper should be considered as a first step
towards a better understanding of the structure of ideals in the ring of
polynomials with an infinite number of indeterminates (up-to a group action).
However, there are several major roadblocks that remain to be solved. The most
important one is the fact that in the presence of the group action, one cannot
use the \kl[hoareDiv]{Hoare divisibility relation} to compare ideals. To understand this
problem, let us assume that a group $\Grp$ acts on the set of indeterminates
$\Indets$. We can define the \reintro[gdiv]{divisibility up-to $\Grp$} relation
$\intro*\gdivleq$ on $\Mon{\Indets}$ as follows: $\mon[M] \gdivleq \mon[M']$ if
there exists $g \in \Grp$ such that $\mon[M] \divleq g \cdot \mon[M']$. One can
safely assume that the ordering $\gdivleq$ is a \kl{well-quasi-order} on
$\Mon{\Indets}$, since otherwise \kl{Hilbert's Basis Theorem} would not hold
\cite{GHOLAS24}.

For instance, let us consider the group $\Grp$ of all permutations of the
indeterminates $\Indets$. In this setting, $x \gdivleq y$ for all
indeterminates $x$ and $y$ in $\Indets$. Let us now consider the polynomial $p
= x + y$ and the polynomial $q = x$. We have $\MaxMon(p) = \set{x, y}$ and
$\MaxMon(q) = \set{x}$, hence $\MaxMon(p) \equiv \MaxMon(q)$ for the Hoare
ordering on the powersets of monomials. This is extremely problematic: one
cannot rely on the equivalence of the maximal monomials to safely perform
substractions and decrease the set of maximal monomials.

The natural ordering on the set of maximal monomials would rather be $S \leq T$
if there exists a $g \in \Grp$ such that $S \hoareDivleq g \cdot T$, that is,
we find a uniform $g$ that works for all monomials, and not one $g$ per
monomial. However, this ordering is never a \kl{well-quasi-order} on the sets
of monomials: the sequence $\seqof{S_i}[i \in \Nat]$ of the sets $S_i \defined
\set{ x_0 x_1, \ldots, x_{i-1} x_i, x_i x_0 }$ is an infinite \kl{antichain}
for this relation, regardless of the group action.
