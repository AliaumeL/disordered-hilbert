% LTeX: language=en
\section{Hilbert's Basis Theorem}
\label{sec:hilbert}

Let $\Field$ be a field, and $\Indets$ a finite set of indeterminates. We
consider the ring of polynomials $\Poly{\Indets}$. We will denote by
$\Mon{\Indets}$ the set of monomials over $\Field$ and $\Indets$, that are
products of indeterminates (without leading coefficient). Note that monomials
are naturally quasi-ordered by divisibility, where $\mon \divleq \mon'$ if
$\mon$ divides $\mon'$. In this setting, let us recall a consequence of
Higman's Lemma \cite{HIG52}: the set of monomials $\Mon{\Indets}$ is
well-quasi-ordered by divisibility.

To a polynomial $p \in \Poly{\Indets}$, we associate the set of maximal
monomials $\MaxMon(p)$, that is the set of monomials that are maximal for the
divisibility relation among the monomials of $p$. Remark that $\MaxMon \colon
\Poly{\Indets} \to \Pfin(\Mon{\Indets})$. Because $\Mon{\Indets}$ is
well-quasi-ordered by divisibility, the set $\Pfin(\Mon{\Indets})$ is also
well-quasi-ordered by the Hoare divisibility relation $\hoareDivleq$: $S \hoareDivleq
S'$ if for all $M \in S$, there exists $M' \in S'$ such that $M \divleq M'$.
We can use this to endow $\Poly{\Indets}$ with a well-quasi-ordering $\pmonleq$:
$p \pmonleq p'$ if $\MaxMon(p) \hoareDivleq \MaxMon(p')$.

One of the main remarks of this section is that one recovers a tight
correspondence between elements of subsets when comparing they upward (resp.
downward) closures.

\begin{lemma}
    \label{lem:antichain-bijection}
    Let $(X,\leq)$ be a quasi-ordered set
    and let $S, S'$ be two finite subsets of $X$
    that are antichains for $\leq$,
    and such that $\upset{S} = \upset{S'}$.
    Then, there exists a bijection $\sigma \colon S \tobij S'$
    such that $s \equiv \sigma(s)$ for all $s \in S$.

    The same result holds under the assumption that
    $\dwset{S} = \dwset{S'}$.
\end{lemma}
\begin{proof}
    Let $x \in S$, by assumption,
    there exists $y \in S'$
    such that $x \leq y$. But we also have
    an $x' \in S$ such that $y \leq x'$.
    Because $S$ is an antichain, we have $x = x'$,
    and therefore $x \equiv y$.
    We can define $\sigma(x) \defined y$.
    Now, let us prove that $\sigma$ is injective.
    Assume that $\sigma(x) = \sigma(x')$.
    Then, $x \equiv \sigma(x) = \sigma(x') \equiv x'$,
    and because $S$ is an antichain, we have $x = x'$.
    Let us now prove that $\sigma$ is surjective.
    This is because $S$ and $S'$ have the same size, that is
    the number of equivalence classes of minimal elements
    in $\upset{S} = \upset{S'}$.
    For the second part of the lemma, we apply
    the first part to the quasi-ordering $\geq$.
\end{proof}


\begin{lemma}
   \label{lem:polynomial-division}
   Let $p, q \in \Poly{\Indets}$ be two non-zero polynomials,
   such that $p \pmoneq q$.
    Then, there exists a constant $c \in \Field$ such that
    $(p - c \times q) \pmonlt q$.
\end{lemma}
\begin{proof}
    Let us first remark that $p \pmoneq q$ if and only if
     $\MaxMon(p) \hoareDivEq \MaxMon(q)$.
    Because $\MaxMon(p)$ and $\MaxMon(q)$ are antichains,
    one can apply \cref{lem:antichain-bijection} to find
    a bijection $\sigma \colon \MaxMon(p) \tobij \MaxMon(q)$
    such that $m \equiv \sigma(m)$ for all $m \in \MaxMon(p)$.
    Now, since the divisibility relation on monomials
    is antisymmetric (recall that monomials do not 
    have multiplicative constants coming from $\Field$), we have that $m = \sigma(m)$
    for all $m \in \MaxMon(p)$. 
    As a consequence, we can write $p = \sum_{m \in \MaxMon(p)} a_m \times m +
    r$ and $q = \sum_{m \in \MaxMon(q)} b_m \times m + r'$, where $a_m, b_m \in
    \Field$ and $R$ and $R'$ are the remaining terms in $p$ and $q$, i.e., are
    such that $r \pmonlt q$ and $r' \pmonlt q$.

    Now, one can select some $m_0 \in \MaxMon(p)$ (which is possible because
    $p$ and $q$ are non-zero polynomials), and write
    $q - c p = \sum_{m \in \MaxMon(q)} (b_m - c \times a_m) m + r' - c \times r$.
    If we select $c = b_{m_0}/ a_{m_0}$ (which is possible because 
    the numbers $a_{m_0}$ and $b_{m_0}$ are non-zero), then we have that 
    $m \notin \MaxMon(q - c \times p)$. However, by construction,
    every maximal monomial of $q - c \times p$ is in $\MaxMon(q)$.
    We conclude that $(q - c \times p) \pmonlt q$.
\end{proof}

\begin{theorem}[Hilbert's Basis Theorem]
    Let $\seqof{\idl_n}$ be an
    ascending chain of ideals in $\Poly{\Indets}$.
    Then, there exists an integer $N$ such that for all $n \geq \Nat$,
    $\idl_n = \idl_N$.
\end{theorem}
\begin{proof}
    Assume towards a contradiction that there is no such
    integer $N$, and without loss of generality, that
    for every $n \in \Nat$, $\idl_n \subsetneq \idl_{n+1}$.
    We can also assume that $\idl_0 \neq \{0\}$.

    Let us define $H_n \defined \upset[\pmonleq]{\idl_n}$ for all $n \in \Nat$.
    Because $\idl_n \subsetneq \idl_{n+1}$, it is clear that $H_n \subseteq
    H_{n+1}$. Because $\Poly{\Indets}$ is well-quasi-ordered by $\pmonleq$,
    every set $H_n$ can be written as $H_n \defined \upset[\pmonleq]{S_n}$ for
    some finite set $S_n \subseteq \Poly{\Indets}$ that is an antichain for
    $\pmonleq$. Let us remark that $0 \notin S_n$ for all $n \in \Nat$, and
    that $S_n$ in non-empty for all $n \in \Nat$.
    Without loss of generality, one can further assume that
    $S_n \subseteq \idl_n$ for all $n \in \Nat$.

    Since, $(\Poly{\Indets}, \pmonleq)$ is well-quasi-ordered,
    there exists an integer $N$ such that 
    $H_N = \bigcup_{n \in \Nat} H_n$.
    In particular, we have $\upset[\pmonleq]{S_N} = \upset[\pmonleq]{S_{N+1}}$.
    Applying \cref{lem:antichain-bijection}, we find a bijection
    $\sigma \colon S_N \tobij S_{N+1}$ such that $p \pmoneq \sigma(p)$
    for all $p \in S_N$.
    Using the fact that $S_N$ is non-empty and that $0 \notin S_N$,
    we can find a non-zero polynomial $p \in S_N$,
    and its corresponding $q \defined \sigma(p) \in S_{N+1}$.

    By \cref{lem:polynomial-division}, we can find a constant $c \in \Field$
    such that $(p - c \times q) \pmonlt q$. Now, because 
    $p \in S_N \subseteq \idl_N \subseteq \idl_{N+1}$ 
    and $q \in S_{N+1} \subseteq \idl_{N+1}$,
    we have that $(p - c \times q) \in \idl_{N+1}$.
    This contradicts the minimality of $q$ in $H_{N+1}$.
\end{proof}
