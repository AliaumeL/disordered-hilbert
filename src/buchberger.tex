% LTeX: language=en
\section{Buchberger's Algorithm}
\label{sec:buchberger}

\AP While the results of \cref{sec:hilbert} show that every \kl{ideal} of
multivariate polynomials is finitely generated, they do not provide a
constructive way to manipulate such sets. In particular, answering the
membership problem for a polynomial $p$ in the ideal generated by a finite set
$H$ of polynomials is not easy. A traditional way to do it is by using the
notion of \kl{Gröbner basis}, which is a finite set of polynomials that
generates the same ideal as $H$ and for which there is a very simple algorithm
to check membership. The simplest algorithm to compute a \kl{Gröbner basis} is
Buchberger's algorithm \cite{BUCH76}, which performs a series of saturating
steps.

\AP Instead of defining the \kl{Gröbner basis} directly, we will first define
the idea behind the membership algorithm. Let us first provide a naïve but
incomplete algorithm for membership testing: to any set $H$ of polynomials, one
can associate a rewriting system $\intro*\premulteucl{H}$ that is representing
a \intro{multiviriate Euclidian division}. A transition of the form $p
\premulteucl{H} r$ means that there exists a \kl{monomial} $\mon$ and a
constant $a \in \Field$ such that $p = a \mon q + r$ for some $q \in H$. It is
easy to see that $p \in \idlGen{H}$ if and only if $p \premulteucl{H}^* 0$. 

\AP However, due to the fact that this rewriting system is reversible, it is
not terminating and does not yield a decision procedure for the $\idlGen{H}$
membership. One way to ensure termination is to select a \kl{well-founded}
\kl{total ordering} $\leq$ polynomials, and restrict the rewriting system to
the transitions that strictly decrease in this ordering. This defines a
sub-transition system $\multeucl{H}{\leq}$ that is no longer reversible by
construction, and is furthermore \intro{terminating}: there are no infinite
sequences of reductions in the system.

\AP The rewriting system $\multeucl{H}{\leq}$ is no longer
\intro(eucl){complete}: there might be polynomials $p$ such that $p \in
\idlGen{H}$ but $p \not\multeucl{H}{\leq}^* 0$. In this abstract
formulation, and having fixed an ordering $\leq$, a \intro{Gröbner basis} of an
\kl{ideal} $\idl$ is a finite set $G$ of polynomials such that $G$ generates
the ideal $\idl$; the rewriting system $\multeucl{G}{\leq}$ is
\kl(eucl){complete} for the ideal $\idl$.

\AP In practice, the ordering $\leq$ is usually constructed from a
\kl{well-founded} \kl{total ordering} on the \kl{monomials}, and extended to
polynomials by comparing their leading monomials (for this ordering). This
total ordering is itself obtained by a choice of an ordering on the
indeterminates $\Indets$. Because we are precisely interested in the
``orderless'' setting, we will cannot follow the classical proofs here.
Instead, we will reuse the quasi-ordering $\pmonleq$ defined in
\cref{sec:hilbert}.

\AP In order to prove that a given rewrite system $\multeucl{H}{\leq}$ is
\kl(eucl){complete}, it is interesting to notice the connection between this
property and a classical notion in rewriting theory, namely \kl{confluence}. A
rewriting system $\to$ is \intro{confluent} whenever, for all triples $x,y,z$
such that $x \to y$ and $x \to z$, there exists a $t$ such that $y \to^* t$ and
$z \to^* t$. Note that, as for the \kl(eucl){completeness}, there is no reason
for a relation $\multeucl{H}{\leq}$ to be \kl{confluent}. In the classical
setting, one can show that \kl{confluence} is equivalent to
\kl(eucl){completeness}.

\AP One particularity of the directed rewrite systems is that when a polynomial
$p$ rewrites to zero, not only do we conclude that $p \in \idlGen{H}$, but also
that $p \in \idlGen{H_{\pmonleq p}}$, where $H_{\pmonleq p}$ is the set of
elements in $H$ that are less or equal to $p$.

\begin{lemma}
  \label{lem:reduction-stronger}
  Let $H$ be a finite set of polynomial
  and 
  let $p$ be a polynomial such that
  $p \multeucl{H}{\pmonleq}^* 0$.
  Then, $p \in \idlGen{H_{\pmonleq p}}$.
\end{lemma}
\begin{proof}
  We prove the result by induction on the length of the
  derivation $p \multeucl{H}{\pmonleq}^* 0$.
  The base case is trivial, because $0 \in \idlGen{\emptyset}$.
  For the inductive step, we assume that 
  $p \multeucl{H}{\pmonleq} r$ for some $r$, 
  and that $r \multeucl{H}{\pmonleq}^* 0$.
  By the inductive hypothesis, we have $r \in \idlGen{H_{\pmonleq r}}
  \subseteq \idlGen{H_{\pmonleq p}}$ since $r \pmonlt p$.
  By construction, 
  there exists a polynomial $h \in H$, a monomial $\mon$ and a constant $a \in \Field$
  such that $p = r + a \mon h$, and $r < p$.
  In particular,
  $\MaxMon(p) = \MaxMon(r + a \mon h)$ and $\MaxMon(r) \hoareDivlt \MaxMon(p)$.
  This means that $\MaxMon(a \mon h) \hoareDivleq \MaxMon(p)$,
  i.e., that $h \pmonleq p$.
  As a consequence, $p \in \idlGen{H_{\pmonleq p}}$.
\end{proof}

Armed with this simple fact, we can already prove that the rewriting system is
\kl(eucl){complete} if it is \kl{confluent}.

\begin{lemma}
    \label{lem:confl-impl-complete}
    Let $H$ be a finite set of polynomials.
    If the rewriting system $\multeucl{H}{\pmonleq}$ is \kl{confluent}
    then it is \kl(eucl){complete}.
\end{lemma}
\begin{proof}
    Assume towards a contradiction that
    there is a polynomial $p \in \idlGen{H}$
    that does not rewrite to $0$.
    Because $\pmonleq$ is a \kl{well-quasi-ordering},
    one can assume $p$ to be minimal for this property.

    Since $p \in \idlGen{H}$,
    there exists a sequence $\seqof{a_i}[1 \leq i \leq n]$
    of non-zero coefficients, 
    a sequence $\seqof{\mon_i}[1 \leq i \leq n]$
    of monomials,
    and a sequence 
    $\seqof{h_i}[1 \leq i \leq n]$ of elements of $H$
    such that:
    \begin{equation}
        p = \sum_{1 \leq i \leq n} a_i \mon_i h_i \quad .
    \end{equation}

    Let us write $Y$ the set of maximal monomials among all monomials
    in the polynomials $a_i \mon_i h_i$, for $1 \leq i \leq n$.
    There are two cases:
    \begin{description}
      \item[The sets $\MaxMon(p)$ and $Y$ are equal.]

        \todo[inline]{write the equation}

      \item[The sets are not equal.]
            
            Let us write $J$ the set of indices $1 \leq j \leq n$
            such that $\mon[L]$ is the maximal monomial
            of $\mon_j h_j$. Let us also assume that the
            (non-zero) coefficient of the monomial $\mon[L]$
            in $\mon_j h_j$ is $1$ (that can be done by changing
            the $a_i$'s).
            Because $\mon[L] \neq \mon[K]$,
            we conclude that
            \begin{equation}
                \sum_{j \in J}
                a_j = 0 \quad .
            \end{equation}
            Because the $a_j$'s are non-zero and
            $J$ is non-empty, we conclude that $J$ contains at least two 
            elements. Let us call
            $J^1$ the set $J$ where one element has been 
            removed. To avoid nested indices, we call this
            element $\star$.

            Let us rewrite the polynomial $p$
            as follows, letting $I = \set{ 1, \dots, n}$:
            \begin{equation}
                p = \sum_{j \in J^1} a_j (\mon_j h_j - \mon_\star h_\star)
                  + 
                    \sum_{i \in I \setminus J} a_i \mon_i h_i
                  \quad .
            \end{equation}

            Now, because the rewriting system is confluent,
            and because
            $\mon[L]$
            can be rewritten into 
            $\mon[L] - \mon_j h_j$
            for all $j \in J$,
            there exists a polynomial $t$ such that
            $\mon[L] - t = 
            \mon_j h_j +  r_j$ where $r_j$ is obtained as a combination of elements 
                    in $H$ having maximal monomials strictly below $\mon[L]$.
            As a consequence,
            \begin{equation*}
                \mon_j h_j - 
                \mon_\star h_\star
                =
                r_\star
                -
                r_j
                \quad .
            \end{equation*}

            We have obtained a new way to write $p$
            using elements of $H$ having a strictly
            smaller maximal monomial. This contradicts
            the minimality of $\mon[L]$.
    \end{description}
\end{proof}

\AP Notice that in the proof of \cref{lem:confl-impl-complete}, we only used
the confluence based on pairs of polynomials. This means that the only property
we need to check is that the rewriting system is \kl{confluent} for specific
pairs of polynomials. 

\begin{lemma}
  The saturation process stops.
\end{lemma}

\begin{lemma}
  Saturated sets are \kl{Gröbner bases}.
\end{lemma}

We have computed the \kl{Gröbner basis} of the ideal generated by
finite set $H$ of polynomials, without using any ordering on the
indeterminates themselves.
