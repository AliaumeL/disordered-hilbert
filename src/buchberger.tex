% LTeX: language=en
\section{Buchberger's Algorithm}
\label{sec:buchberger}

In this section, we present a version of Buchberger's algorithm that does not
require a total ordering on the indeterminates. This will generalise the proof
of Hilbert's Basis Theorem done in \cref{sec:hilbert}. Let us first recall the
intuition behind Buchberger's algorithm, and the actual problem it is solving.

Given a finite set $H$ of polynomials, and a polynomial $p$, we want to check
whether $p$ is in the ideal generated by $H$. More generally, we would like to
have a finite representation of the ideal generated by $H$ over which it is
easy to compute such membership queries, unions, intersections, etc. Note that
the membership problem is generalises reachability and is therefore EXPSPACE.
On more efficient representations, called \kl{Gröbner bases}, the same
problem is in lintime. In this setting, Buchberger's algorithm is a
precomputation procedure.

Let us first provide a naïve but incomplete algorithm for membership testing:
to any set $H$ of polynomials, one can associate a rewriting system
$\intro*\multeucl{H}$ that is representing \emph{multiviriate Euclidian
division}. A transition of the form $p \multeucl{H} r$ means that there exists
a \kl{monomial} $\mon$ and a constant $a$ such that $p = a \mon q + r$ for some
$q \in H$. It is easy to see that $p \in \idlGen{H}$ if and only if $p
\multeucl{H}^* 0$. 

However, there is no guarantee that this rewriting system is terminating. One
solution to ensure termination is to select a well-founded total ordering on
monomials that is compatible with the multiplicative structure, and restrict
the rewriting system to the transitions that strictly decrease in this
ordering. The existence of such an ordering is the easy part of the proof, but
it creates a problem: it is not clear that the rewriting system is complete. It
turns out that completeness is equivalent to confluence [todo].

\AP
A \intro{monoidal ordering} is ...

\begin{lemma}
    For any finite set of indeterminates there exists
    a well-founded monoidal ordering.
\end{lemma}

\begin{lemma}
    Let $H$ be a finite set of polynomials
    and $<$ be a well-founded monoidal ordering.
    The rewriting system $\multeucl{H}$ is confluent
    if and only if it is complete, i.e. for all
    $p \in \idlGen{H}$, $p \multeucl{H}^* 0$.
\end{lemma}

\begin{figure}
    \centering
    \begin{tikzpicture}
        \node (p) {$xyz$};
        \node[below=of p] (r1) {$y$};
        \node[right=of p] (r2) {$z$};

        \path[-stealth]
            (p) edge node[left]{$\multeucl{H}$} (r1)
            (p) edge node[midway, above]{$\multeucl{H}$} (r2);
    \end{tikzpicture}
    \caption{
        An example of non confluence
        for the rewriting system
        $\multeucl{H}$, where $H = \set{ xy, xz }$,
        for any choice of \kl{monoidal ordering}
        $<$.
    }
    \label{fig:buchberger-non-confluent}
\end{figure}

The classical method to ensure confluence is to 
understand the \intro{critical pairs} of the rewriting
system, which essentially amounts to

\subsection{Orderless Buchberger}

The only idea here is to not use a \kl{monoidal ordering} but
to use a \kl{divisibility partial ordering} instead. Note that because
the latter in not total, one has to be careful about 
the precise statements. However, we will show that if one
orders polynomials according to their set of maximal monomials
(with the powerset ordering), then one can redo the confluence analysis
to obtain a complete rewriting system.


