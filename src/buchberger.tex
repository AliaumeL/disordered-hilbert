% LTeX: language=en
\section{Buchberger's Algorithm}
\label{sec:buchberger}

In this section, we present a version of Buchberger's algorithm that does not
require a total ordering on the indeterminates. This will generalise the proof
of Hilbert's Basis Theorem done in \cref{sec:hilbert}. Let us first recall the
intuition behind Buchberger's algorithm, and the actual problem it is solving.

Given a finite set $H$ of polynomials, and a polynomial $p$, we want to check
whether $p$ is in the ideal generated by $H$. More generally, we would like to
have a finite representation of the ideal generated by $H$ over which it is
easy to compute such membership queries, unions, intersections, etc. Note that
the membership problem is generalises reachability and is therefore EXPSPACE.
On more efficient representations, called \kl{Gröbner bases}, the same
problem is in lintime. In this setting, Buchberger's algorithm is a
precomputation procedure.

\AP Let us first provide a naïve but incomplete algorithm for membership
testing: to any set $H$ of polynomials, one can associate a rewriting system
$\intro*\premulteucl{H}$ that is representing \emph{multiviriate Euclidian
division}. A transition of the form $p \premulteucl{H} r$ means that there
exists a \kl{monomial} $\mon$ and a constant $a$ such that $p = a \mon q + r$
for some $q \in H$. It is easy to see that $p \in \idlGen{H}$ if and only if $p
\premulteucl{H}^* 0$. 

\AP
However, due to the fact that this rewriting system is reversible, it is not
terminating and does not yield a decision procedure for the $\idlGen{H}$
membership. One way to ensure termination is to select a \kl{well-founded}
\kl{total ordering} $\monord$ on monomials that is \kl{compatible with the multiplicative
structure}, and restrict the rewriting system to the transitions that strictly
decrease in this ordering. This defines
a sub-transition system $\multeucl{H}{\monord}$ that is no longer reversible
by construction, and is furthermore \intro{terminating}: there are no
infinite sequences of reductions in the system.

\AP A (pre)order $\leq$ is a \intro{monomial ordering}
if for all monomials $\mon_1, \mon_2, \mon_3$, such that $\mon_1 \leq \mon_2$,
$\mon_1 \mon_3 \leq \mon_2 \mon_3$. The existence of \kl{well-founded}
\kl(order){total} \kl{monomial orderings} follows from a relatively
straightforward result in order theory: for every $k \in \Nat$, $\Nat^k$ is a
\kl{well-quasi-order} when endowed with the componentwise ordering.


\begin{lemma}[Folklore]
    \label{lem:monomial-order-exists}
    For every finite set $\Indets$ of indeterminates,
    there exists a \kl{well-founded}
    \kl(order){total} \kl{monomial orderings}. Furthermore,
    it is computable.
\end{lemma}
\begin{proof}
    Select any total ordering $<$ on the finite set of
    indeterminates $\Indets$. We define our total ordering
    as the lexicographic ordering obtained when considering 
    monomials as sequences of numbers.
    It is clearly a \kl(order){total}
    \kl{monomial ordering}.
    Because of Dickson's Lemma \cite{SCSC12},
    the set of monomials endowed with the \kl{divisibility preordering}
    is a \kl{well-quasi-order}.
    Since our ordering is a linearisation of the 
    \kl{divisibility preordering} (that is, if
    $\mon_1 \divleq \mon_2$, then $\mon_1 \leq \mon_2$),
    it must therefore be \kl{well-founded}
    \cite{SCSC12}.
\end{proof}

\AP In order to prove that a given rewrite system $\multeucl{H}{\leq}$ is
\kl(eucl){complete}, it is interesting to notice the connection between this
property and a classical notion in rewriting theory, namely \kl{confluence}. A
rewriting system $\to$ is \intro{confluent} whenever, for all triples $x,y,z$
such that $x \to y$ and $x \to z$, there exists a $t$ such that $y \to^* t$ and
$z \to^* t$. Note that, as for the \kl(eucl){completeness}, there is no reason
for a relation $\multeucl{H}{\leq}$ to be \kl{confluent} (see
\cref{fig:buchberger-non-confluent}).

\begin{figure}
    \centering
    \begin{tikzpicture}
        \node (p) {$xyz$};
        \node[below=of p] (r1) {$y$};
        \node[right=of p] (r2) {$z$};

        \path[-stealth]
        (p) edge node[font=\tiny,left]{$(H,\leq)$} (r1)
        (p) edge node[font=\tiny, midway, above]{$(H,\leq)$} (r2);
    \end{tikzpicture}
    \caption{
        An example of non confluence
        for the rewriting system
        $\multeucl{H}{\leq}$, where $H = \set{ xy, xz }$,
        for any choice of \kl{monoidal ordering}
        $\leq$.
    }
    \label{fig:buchberger-non-confluent}
\end{figure}

One particularity of the directed rewrite systems is that when a polynomial $p$
rewrites to zero, not only do we conclude that $p \in \idlGen{H}$, but also
that $p \in \idlGen{H_{\leq p}}$, where $H_{\leq p}$ is the set of elements in
$H$ that are less or equal to $p$.

\begin{lemma}
    \label{lem:confl-impl-complete}
    Let $H$ be a finite set of polynomials
    and $\leq$ be a \kl{well-founded} \kl(order){total}
    \kl{monomial ordering}.
    If the rewriting system $\multeucl{H}{\leq}$ is \kl{confluent}
    then it is \kl(eucl){complete}.
\end{lemma}
\begin{proof}
    Assume towards a contradiction that
    there is a polynomial $p \in \idlGen{H}$
    that does not rewrite to $0$.
    Because $\leq$ is a \kl{total} and \kl{well-founded},
    one can assume $p$ to be minimal for this property.

    Since $p \in \idlGen{H}$,
    there exists a sequence $\seqof{a_i}[1 \leq i \leq n]$
    of non-zero coefficients, 
    a sequence $\seqof{\mon_i}[1 \leq i \leq n]$
    of monomials,
    and a sequence 
    $\seqof{h_i}[1 \leq i \leq n]$ of elements of $H$
    such that:
    \begin{equation}
        p = \sum_{1 \leq i \leq n} a_i \mon_i h_i \quad .
    \end{equation}

    Let $\mon[K]$ be maximal among the monomials of $p$,
    and $\mon[L]$ be maximal among the monomials
    of the polynomials $\mon_i h_i$ for $1 \leq i \leq n$.
    Without loss of generality, one can also assume that 
    $\mon[L]$ is minimal among all representations of $p$.
    There are two cases:

    \begin{description}
        \item[The two monomials are equal.]
           In this case, $\mon[K]$ is the maximal monomial of
           \emph{some} $\mon_i h_i$.
           Then, let us write $\beta$ for the coefficient of
           $\mon[K]$ in $p$,
           and
           $r \defined p - \beta \mon_i h_i$.
           It is clear that 
           $r < p$, and that $p \premulteucl{H} r$,
           hence that 
           $p \multeucl{H}{\leq} r$.

           We conclude that $r \in \idlGen{H}$, 
           and by minimality of $p$, $r \multeucl{H}{\leq}^* 0$.
           This proves that 
           $p \multeucl{H}{\leq}^* 0$ 
           which is absurd.

        \item[The two monomials are not equal.]
            
            Let us write $J$ the set of indices $1 \leq j \leq n$
            such that $\mon[L]$ is the maximal monomial
            of $\mon_j h_j$. Let us also assume that the
            (non-zero) coefficient of the monomial $\mon[L]$
            in $\mon_j h_j$ is $1$ (that can be done by changing
            the $a_i$'s).
            Because $\mon[L] \neq \mon[K]$,
            we conclude that
            \begin{equation}
                \sum_{j \in J}
                a_j = 0 \quad .
            \end{equation}
            Because the $a_j$'s are non-zero and
            $J$ is non-empty, we conclude that $J$ contains at least two 
            elements. Let us call
            $J^1$ the set $J$ where one element has been 
            removed. To avoid nested indices, we call this
            element $\star$.

            Let us rewrite the polynomial $p$
            as follows, letting $I = \set{ 1, \dots, n}$:
            \begin{equation}
                p = \sum_{j \in J^1} a_j (\mon_j h_j - \mon_\star h_\star)
                  + 
                    \sum_{i \in I \setminus J} a_i \mon_i h_i
                  \quad .
            \end{equation}

            Now, because the rewriting system is confluent,
            and because
            $\mon[L]$
            can be rewritten into 
            $\mon[L] - \mon_j h_j$
            for all $j \in J$,
            there exists a polynomial $t$ such that
            $\mon[L] - t = 
            \mon_j h_j +  r_j$ where $r_j$ is obtained as a combination of elements 
                    in $H$ having maximal monomials strictly below $\mon[L]$.
            As a consequence,
            \begin{equation*}
                \mon_j h_j - 
                \mon_\star h_\star
                =
                r_\star
                -
                r_j
                \quad .
            \end{equation*}

            We have obtained a new way to write $p$
            using elements of $H$ having a strictly
            smaller maximal monomial. This contradicts
            the minimality of $\mon[L]$.
    \end{description}
\end{proof}

Let us prove the converse direction, i.e. that \kl(eucl){completeness}
implies \kl{confluence}. This will be crucial in understanding the
Buchberger algorithm.

\begin{lemma}
    \label{lem:confl-iff-corr}
    Let $H$ be a finite set of polynomials and 
    $\leq$ be a \kl{well-founded}
    \kl(order){total}
    \kl{monomial ordering}. Then, the following are equivalent:
    \begin{enumerate}
        \item \label{item:complete}
            The rewriting system $\multeucl{H}{\leq}$ is
            \kl(eucl){complete}.
        \item \label{item:confl}
            The rewriting system $\multeucl{H}{\leq}$ is
            \kl{confluent}.
        \item \label{item:wconfl}
            The rewriting system $\multeucl{H}{\leq}$ is
            \kl{confluent} when starting from monomials.
    \end{enumerate}
\end{lemma}
\begin{proof}
    It is clear that \cref{item:confl}
    implies \cref{item:wconfl}. Furthermore,
    \cref{lem:confl-impl-complete} actually only used the
    confluence when starting from monomials,
    effectively proving \cref{item:wconfl}
    implies \cref{item:complete}.

\end{proof}

A consequence of \cref{lem:confl-impl-complete} is that the membership problem
becomes very easy to solve: keep reducing until no more steps can be taken, and
check if the normal form is $0$.

The classical method to ensure confluence is to 
understand the \intro{critical pairs} of the rewriting
system, which essentially amounts to

\subsection{Orderless Buchberger}

The only idea here is to not use a \kl{monoidal ordering} but
to use a \kl{divisibility partial ordering} instead. Note that because
the latter in not total, one has to be careful about 
the precise statements. However, we will show that if one
orders polynomials according to their set of maximal monomials
(with the powerset ordering), then one can redo the confluence analysis
to obtain a complete rewriting system.

\subsection{Atomic Buchberger?}
