% LTeX: language=en
\section{Orbit Finite Computations}
\label{sec:orbit-finite-computations}

In this section we will discuss how the theorems obtained in
\cref{sec:hilbert,sec:buchberger}
can
be used in the case where the set of indeterminates is infinite and equipped
with a group action. Before that, we will need to introduce some notation and
definitions regarding these so-called \emph{orbit finite} sets.

\AP Let us fix a $\Indets$ of indeterminates and a group $\Grp$ acting on
$\Indets$. The action of $\Grp$ on $\Indets$ induces an action on the
polynomial ring $\Poly{\Indets}$, on the sets of monomials $\Mon{\Indets}$, on
sets of monomials $\Pset(\Mon{\Indets})$, and so on. A function $f \colon X \to Y$ is
\intro(function){equivariant} if for all $\gelem \in \Grp$ and $x \in X$, we have $f(\gelem \cdot
x) = \gelem \cdot f(x)$, that is, the function commutes with the action of the group
on the domain and codomain.

\begin{example}
  \label{ex:equivariant}
  Polynomial addition and products are \kl{equivariant} by definition.
  The function $\MaxMon \colon \Poly{\Indets} \to \Pset(\Mon{\Indets})$
  is \kl{equivariant} with respect to the action of $\Grp$ on
  $\Indets$. 
  Similarly, the function $\MaxMonMult \colon \Pset(\Poly{\Indets}) \to
  \Pset(\Mon{\Indets})$ is \kl{equivariant}.
\end{example}

\AP An \intro{equivariant subset} of a set $X$ is a subset $Y \subseteq X$ such
that $\Grp \cdot Y = Y$, that is, for all $\gelem \in \Grp$ and $y \in Y$, we have
$\gelem \cdot y \in Y$. We will denote by $\Orbit{\Grp}{X}$ the set of orbits of $X$
with respect to the action of $\Grp$ on $X$. When the set $\Orbit{\Grp}{X}$ is
a finite set we say that $X$ is \intro{orbit finite} with respect to the action
of $\Grp$ on $X$.




