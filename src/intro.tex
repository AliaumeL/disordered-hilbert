\section{Introduction}
\label{sec:intro}

\AP Let us fix a finite set of indeterminates $\Indets$, a field $\Field$, and
let us consider the ring of polynomials $\Poly{\Indets}$. Important subsets of
a ring in commutative algebra include the \intro{(bilateral) ideals}, which are
subsets of the ring that are closed under addition and multiplication by any
element of the ring. Formally, an ideal $\idl$ of a commutative ring $R$ is a
subset of $R$ such that: for all $p,q \in \idl$ and $r \in R$, $r p + q \in
\idl$. A fundamental result from commutative algebra is \intro{Hilbert's Basis
Theorem}, which states that the ring of polynomials $\Poly{\Indets}$ is
\intro{Noetherian}, that is, every increasing chain of ideals in
$\Poly{\Indets}$ stabilises (see for instance \cite[Theorem 4]{CLO15}). An
equivalent formulation of this theorem is that every \kl{ideal} of polynomials
is \intro{finitely generated}, that is, there exists a finite set of
polynomials $p_1, \ldots, p_n$ such that $\idl = \idlGen{p_1, \ldots, p_n}$,
where $\intro*\idlGen{p_1, \ldots, p_n}$ is the smallest ideal containing $p_1,
\ldots, p_n$.

\AP Because every \kl{ideal} is \kl{finitely generated}, one can try to devise
algorithms to manipulate ideals of polynomials. One of the simplest algorithmic
question is to decide the \intro{ideal membership problem}, that is, given a
polynomial $p$ and an ideal $\idl$, decide whether $p \in \idl$. This problem
is known to be \EXPSPACE-complete \cite{MAME82}, and most of the algorithms
that are used in practice start by computing so-called \emph{Gröbner bases},
that represent the same ideal, but on which the ideal membership problem can be
more efficiently decided, by computing a linear number of polynomial
subtractions in the size of the Gröbner basis. Computing a Gröbner basis can
also be thought of a more algorithmically tractable presentation of \kl{ideals}
in general: not only \kl{membership} can be decided efficiently, but also
inclusions, computation of intersections and so on.

\AP Usually, the first algorithm that one encounters in the context of
\kl{Gröbner bases} is the one of \intro{Buchberger} \cite{BUCH76}, that
computes a Gröbner basis of an ideal $\idl$ from a finite set of its
gererators, and that is fundamentally based on three ingredients: a
\emph{multivariate} analogue of the Euclidean division algorithm,\footnote{
Recall that the ring of polynomials $\Poly{\Indets}$ is not a Euclidean domain
as soon as the number of indeterminates is greater than one.} a saturation
algorithm that bears some resemblance with the Knuth-Bendix completion
algorithm \cite{KNBE70}, and a termination proof based on a well-quasi-ordering
argument.

\AP There have been several attempts to adapt \kl{Hilbert's Basis Theorem} and
\kl{Buchberger's algorithm} to the case of an infinite number of
indeterminates. While the results fail in this case,\footnote{Gonsider the
ideal generated by all the indeterminates, it cannot be finitely generated} one
can still obtain interesting results when the set of indeterminates is equipped
with a group action. For instance, if all the indeterminates are considered to
be indistinguishable (that is, the group action consists of all permutations of
the indeterminates), then one recovers the \emph{up-to} version of
\kl{Hilbert's Basis Theorem} \cite{BRDR11,HIKRLE18}: every increasing chain of
\kl{ideals} (up-to the group action) stabilises. For instance, the ideal
generated by all the indeterminates is generated by a single one of them, up-to
the permutations of variables.

\AP Such approaches to better understand how these theorems can be adapted to
the presence of symmetries are interesting from a mathematical point of view,
but they also have a practical impact, for instance in the verification of
finite systems that manipulate abstract \emph{data} \cite{KAFR94}. 
In this more algorithmic context, the study of vector spaces up to a 
group action showed the pertinence of this approach \cite{BOZLMO21}.
Recently, the authors of \cite{GHOLAS24} proposed a rather general 
approach to the study of \kl{ideals} up to a group action, generalising
the results of \cite{BRDR11,HIKRLE18}.

\AP At the time of writing this document, the best approaches to the study of
\kl{ideals} up to a group actions deeply rely on the existence of a total
ordering on the indeterminates that is compatible with the group action. This
is for instance the case if variables are elements of $\Rat$, and the group is
composed of all monotone bijections of $\Rat$. However, this hypothesis is not
always satisfied, and in particular, it is not satisfied in the context of all
permutations (where one can use a trick to construct a total ordering on the
indeterminates by changing the group action). We believe that relying on the
existence of a total ordering on the indeterminates is an atavism, that
perdures because every proof of \kl{Hilbert's Basis Theorem} and
\kl{Buchberger's algorithm} that we know of starts by picking a total ordering
on the (finite set) of indeterminates, an action that is harmless in the finite
setting, and harmful in the infinite setting.

\paragraph{Contributions} \AP We provide a new self contained proof of
\kl{Hilbert's Basis Theorem} and an analogue of \kl{Buchberger's algorithm}
that do not rely on a total ordering of the indeterminates, in the case of a
finite number of indeterminates. Note that none of the results are new
\emph{per se}, but they focus on a point of view that was mostly ignored for
several reasons: first, in the finite case, the existence of a total ordering
is not harmful and simplifies the proofs; second, the results are deeply
non-constructive and rely multiple times on excluded middle and the axiom of
choice, and removing the need for a total ordering on the indeterminates is
somehow the least concern for constructivists.

\paragraph{Outline} \AP In
\cref{sec:hilbert}, we recall the
statement of \kl{Hilbert's Basis Theorem} and prove it without relying on a
total ordering of the indeterminates. In
\cref{alg:buchberger},
we present an alternative version of \kl{Buchberger's algorithm} that does not
rely on a total ordering of the indeterminates. Finally, in
\cref{sec:conclusion}, we discuss the main
limitations of our approach in the case of an infinite number of
indeterminates.

