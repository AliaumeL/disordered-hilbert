\section{Introduction}
\label{sec:intro}

\AP Let us fix a finite set of indeterminates $\Indets$, a field $\Field$, and
let us consider the ring of polynomials $\Poly{\Indets}$. Important subsets of
a ring in commutative algebra include the \intro{(bilateral) ideals}, which are
subsets of the ring that are closed under addition and multiplication by any
element of the ring. Formally, an ideal $\idl$ of a commutative ring $R$ is a
subset of $R$ such that: for all $p,q \in \idl$ and $r \in R$, $r p + q \in
\idl$. A fundamental result from commutative algebra is \intro{Hilbert's Basis
Theorem}, which states that the ring of polynomials $\Poly{\Indets}$ is
\intro{Noetherian}, that is, every increasing chain of ideals in
$\Poly{\Indets}$ stabilises (see for instance \cite[Theorem 4]{CLO15}). An
equivalent formulation of this theorem is that every \kl{ideal} of polynomials
is \intro{finitely generated}, that is, there exists a finite set of
polynomials $p_1, \ldots, p_n$ such that $\idl = \idlGen{p_1, \ldots, p_n}$,
where $\intro*\idlGen{p_1, \ldots, p_n}$ is the smallest ideal containing $p_1,
\ldots, p_n$.

\AP Because every \kl{ideal} is \kl{finitely generated}, one can try to devise
algorithms to manipulate ideals of polynomials. One of the simplest algorithmic
question is to decide the \intro{ideal membership problem}, that is, given a
polynomial $p$ and an ideal $\idl$, decide whether $p \in \idl$. This problem
is known to be \EXPSPACE-complete \cite{MAME82}, and most of the algorithms
that are used in practice start by computing so-called \emph{Gröbner bases}, 
that represent the same ideal, but on which the ideal membership problem 
can be more efficiently decided, 
that represent the same ideal, but on which the ideal membership problem 
can be more efficiently decided:


\begin{itemize}
    \item Hilbert is a fundamental theorem.
    \item adaptations to new settings (infinite number of
        variables, etc.)
    \item computational aspects (Buchberger, etc.)
    \item One key limitation in new settings is the need
        for a total ordering on the indeterminates.
        (and even then, results are non trivial)
\end{itemize}

\paragraph{Contributions}

New simple and self-contained proof of Hilbert's Basis Theorem
for the case of a finite number of variables, 
without the need for a total ordering on the indeterminates.

Also, adaptation of Buchberger's algorithm to the case of a
finite number of variables, without the need for a total ordering
on the indeterminates.

\paragraph{Outline}
