%! TEX program = pdflatex
% WARNING: this is a generated file.
%
% Please do not edit this file directly. 
% - If you want to update the medatata of the paper (title, authors, abstract), please
%   edit the `paper-meta.yaml` file in the root of the repository.
% - If you want to update the content of the paper, please edit the latex files
%   in the `src` directory.
% - If you want to update the template itself (e.g., change the layout), please
%   edit the `templates/plain-article.tex` file instead.
\documentclass[11pt,a4paper,twosided]{article}

% we setup a custom geometry because the default one is too narrow
\usepackage{geometry}
\geometry{margin=3.5cm}

% utf-8 for old systems
\usepackage[utf8]{inputenc}
\usepackage[T1]{fontenc}

% babel for language settings
\usepackage[english]{babel}

% microtype for better typography
\usepackage{microtype}

\usepackage{todonotes}
\usepackage{lineno}


% math packages
\usepackage{amsmath,amsthm,amssymb,stmaryrd,thmtools,upgreek}

% configure some theorems
\newtheorem{theorem}{Theorem}
\newtheorem{lemma}[theorem]{Lemma}
\newtheorem{corollary}[theorem]{Corollary}
\newtheorem{proposition}[theorem]{Proposition}
\newtheorem{conjecture}[theorem]{Conjecture}
\theoremstyle{definition}
\newtheorem{definition}[theorem]{Definition}
\newtheorem{remark}[theorem]{Remark}
\newtheorem{example}[theorem]{Example}


% graphics packages
\usepackage{graphicx}
\usepackage[obeyclassoptions,mode=tex]{standalone}
\usepackage{tikz}
\usetikzlibrary{backgrounds}
\usetikzlibrary{shapes.geometric}
\usetikzlibrary{positioning}
\usetikzlibrary{automata}
\usetikzlibrary{tikzmark}
\usetikzlibrary{patterns}
\usetikzlibrary{arrows}
\tikzset{every state/.style={minimum size=1pt}}
\usepackage{tikz-cd}

% ornaments
\usepackage{pgfornament}


% links inside the document
\usepackage{hyperref}
\usepackage[capitalise,noabbrev,nameinlink]{cleveref}
\usepackage[composition,hyperref,xcolor]{knowledge}
\knowledgeconfigure{notion}

% Tables 
\usepackage{booktabs}
\usepackage{varwidth}

% Algorithms
\usepackage{algorithm2e}
\Crefname{algocfline}{Algorithm}{Algorithms}
\crefname{algocfline}{Algorithm}{Algorithms}
\crefname{algocf}{Algorithm}{Algorithms}
\Crefname{algocf}{Algorithm}{Algorithms}

% Packages for macro definitions
\usepackage{xparse}
\usepackage{xpatch}
\usepackage{tokcycle}
\usepackage{ifthen}

% Proofs
\usepackage{bussproofs}

% Colors 
\usepackage{ensps-colorscheme}
% parametrize knowledge colors and style
% intro => A4 + emph 
% kl    => normal color, normal size not emph
\knowledgestyle{intro notion}{color={A2}, emphasize}
\knowledgestyle{notion}{color={B1}}
\hypersetup{
    colorlinks=true,
    breaklinks=true,
    linkcolor=A4,
    citecolor=A4,
    urlcolor=A4,
    filecolor=A4,
}


% we include whatever the user wants to include in the header

% we include libraries (tex files) usually written in the `lib` directory

% Knowledge logo
\newcommand{\klogo}{%
\begin{tikzpicture}[scale=0.2,line/.style={draw, line width=0.2pt, line cap=round, line join=round}]
\coordinate (A00) at (0,0);
\coordinate (A01) at (0,1);
\coordinate (A10) at (1,0);
\coordinate (B10) at (1,0.2);
\coordinate (B01) at (0.2,1);

\coordinate (C01) at (0.4,0.7);
\coordinate (C10) at (0.7,0.4);
\coordinate (C12) at (0.4,1.2);
\coordinate (C21) at (1.2, 0.4);
\coordinate (C22) at (1.2, 1.2);

\coordinate (D00) at (C10);
\coordinate (D01) at (0.8,0.5);
\coordinate (D10) at (0.8,0.3);

\coordinate (E01) at (0.3,0.7);
\coordinate (E10) at (0.5,0.7);

\draw[line] (B01) -- (A01) -- (A00) -- (A10) -- (B10);
\draw[line] (C01) -- (C12) -- (C22) -- (C21) -- (C10);

\draw[line] (D01) -- (D00) -- (D10);
\draw[line] (E01) -- (E10);

\end{tikzpicture}%
}

% Upgreek letters
\makeatletter
\newcommand\mathgr[1]{\tokcycle
  {\addcytoks{##1}}
  {\processtoks{##1}}
  {\ifcsname up\expandafter\@gobble\string##1\endcsname
   \addcytoks[1]{\csname up\expandafter\@gobble\string##1\endcsname}%
    \else\addcytoks{##1}\fi}
  {\addcytoks{##1}}{#1}%
  \expandafter\mathrm\expandafter{\the\cytoks}%
}
\makeatother


% Create a new macro proofof
% taking as input a label of a theorem
% and creating a proof with a reference to that
% label
\NewDocumentEnvironment{proofof}{ m O{appendix} }{
    % if the command \#1 exists, then 
    % call \#1* to restate the theorem
    \ifcsname #1\endcsname
        \def\isInsideRestatedTheorem{1}
        \csname #1\endcsname*
    \fi
    \begin{proof}[Proof of {\cref{#1}} as stated on page {\pageref{#1}}]
        \phantomsection
        \label{#1:proof}
}{
        % if the optional argument is "appendix" 
        % then printout a "backlink"
        % and otherwise do nothing
        \ifthenelse{\equal{#2}{appendix}}{
        % Some link to go back to the theorem
        \marginpar{\vspace{-2em}\texttt{\small{\hyperref[#1]{$\triangleright$ Back to p.\pageref{#1}}}}}
        }{}
    \end{proof}
}

% Create a new macro proofref
% that takes as input a label of a theorem
% and creates a reference to its proof
\NewDocumentCommand{\proofref}{ m }{
    % checks if the label #1:proof exists, if yes
    % it creates a link to it, otherwise it writes nothing
    \IfRefUndefinedExpandable{#1:proof}{}{
        % Checks if we are inside a restated theorem
        % if yes, we do not print anything
        \ifdefined\isInsideRestatedTheorem
        \else
            \marginpar{\vspace{0.6em}\texttt{\small{\hyperref[#1:proof]{$\triangleright$ Proven p.\pageref{#1:proof}}}}}
        \fi
    }
}



\newcommand{\circled}[2]{%
\hypertarget{#2}{}%
\tikz[baseline=(char.base),color=A2,thick]{%
\node[shape=circle,draw,inner sep=1pt,font=\tiny] (char) {#1};%
}}
\newcommand{\circleref}[2]{%
\hyperlink{#2}{%
\tikz[baseline=(char.base),color=A2,thick]{%
\node[shape=circle,draw,inner sep=1pt,font=\tiny] (char) {#1};}%
}}

% Little math macros
\NewDocumentCommand{\set}{ m }{\{ #1 \}}
\NewDocumentCommand{\setof}{ m m }{\{ #1 \mid #2 \}}
\NewDocumentCommand{\card}{ m }{\left| #1 \right|}
\NewDocumentCommand{\seqof}{ m O{n \in \Nat} }{\left( #1 \right)_{#2}}

\NewDocumentCommand{\defined}{ }{\triangleq}
\newcommand{\defiff}{\overset{\mathrm{def}}{\iff}}
\newcommand{\defeq}{\overset{\mathrm{def}}{=}}

\newcommand{\subfin}{\subset_{\text{fin}}}

\newcommand{\Pfin}{\mathcal{P}_{\text{fin}}}
\newcommand{\Pset}{\mathcal{P}}

% functions of all sorts (injective, partial, surjective)
\newcommand{\topartial}{\rightharpoonup}
\newcommand{\toinj}{\hookrightarrow}
\newcommand{\tosurj}{\twoheadrightarrow}
\newcommand{\tobij}{\stackrel{\simeq}{\longrightarrow}}

% classical sets
\newcommand{\Nat}{\mathbb{N}}

% Automate the creation of new orderings
% based on a given symbol.
% For instance,
% \NewDocumentOrdering{\pref}{\preceq}{\prec}
% will create the following commands:
% \prefleq and \preflt
% that will respectively expand to
% \mathrel{\kl[\pref]{\preceq}} and \mathrel{\kl[\pref]{\prec}}
\NewDocumentCommand{\NewDocumentOrdering}{ m m m }{
    \expandafter\newcommand\csname #1leq\endcsname{
        \mathrel{\kl[#1]{#2}}
    }
    \expandafter\newcommand\csname #1lt\endcsname{
        \mathrel{\kl[#1]{#3}}
    }
    \knowledge{#1}{notion}
}

% Order macros
\NewDocumentCommand{\upset}{ O{} m }{{\uparrow_{#1} #2}}
\NewDocumentCommand{\dwset}{ O{} m }{{\downarrow_{#1} #2}}


% Number theory
\NewDocumentCommand{\factorial}{ O{} m }{
    \if\relax\detokenize{#1}\relax
        #2!
    \else
        (#2)!
    \fi
}

% orders 
\NewDocumentOrdering{div}{\sqsubseteq^{\mathrm{div}}}{\sqsubset^{\mathrm{div}}}

\NewDocumentOrdering{pmon}{\preceq}{\prec}
\NewDocumentCommand{\pmoneq}{}{\mathrel{\kl[pmon]{\simeq}}}

\NewDocumentOrdering{hoareDiv}{\sqsubseteq^{\mathrm{div}}_{\flat}}{\sqsubset^{\mathrm{div}}_{\flat}}
\NewDocumentCommand{\hoareDivEq}{}{ \mathrel{\kl[\hoareDiv]{\equiv^{\mathrm{div}}_{\flat}}} }

\newcommand{\Field}{\mathbb{K}}
\newcommand{\Indets}{\mathcal{X}}
\NewDocumentCommand{\Poly}{ O{\Field} m }{#1[{#2}]}
\NewDocumentCommand{\Mon}{ O{\Field} m }{\mathsf{Mon}_{#1}({#2})}

\NewDocumentCommand{\mon}{ O{M} }{#1}
\NewDocumentCommand{\pol}{ O{p} }{#1}


\newcommand{\idlGen}[1]{\langle #1\rangle}
\newcommand{\idl}{\mathcal{I}}

\newcommand{\MaxMon}{\mathsf{MaxMon}}



\input{lib/knowledges.kl}

% We include the title and author information based on the 
% `paper-meta.yaml` file.
 
\title{Hilbert's Theorem and Gröbner Bases Without Ordering Indeterminates}

\author{
Aliaume Lopez\thanks{University of Warsaw, Poland. Aliaume Lopez was supported by the Polish National Science Centre (NCN) grant ``Polynomial finite state computation'\,' (2022/46/A/ST6/00072).}
 \and
Arka Ghosh\thanks{Université de Bordeaux, France. TODO arka}
}

% For the date, we first check if the user has provided a date,
% and otherwise use the git meta inforamtion (if available).
\date{2025-03-21 16:25:43 +0100\footnote{79e9c47117e41db785ea753ca7279dc1d2151cf2 -- branch main at git@github.com:AliaumeL/disordered-hilbert.git}}

\newcommand{\repositoryUrl}{\url{https://github.com/AliaumeL/disordered-hilbert}}

\knowledge{notion}
 | kl-usage

% Now, we create the document itself.
\begin{document}
% Generate the title page
\maketitle
% Print the abstract
\begin{abstract}
    Classical proofs of Hilbert's basis theorem, stating that every ideal in a polynomial ring over a field has a finite generating set, rely on picking a total ordering on the indeterminites. This need for a total ordering is even more pregnant in the algorithmic setting, where the decision procedures for the ideal membership problem (such as the computation of Gröbner bases) crucially depend on the choice of ordering on the indeterminates. In this paper, we provide a new proof of Hilbert's basis theorem that does not depend on a total ordering on the indeterminates, and adapt the Buchberger algorithm to compute Gröbner bases in this setting.
    \paragraph{Keywords:}
    Hilbert basis, ideal membership problem, well-quasi-ordering, Buchberger algorithm, Gröbner bases
\paragraph{Repository:} \repositoryUrl
\end{abstract}

\klogo\ This document uses \href{https://ctan.org/pkg/knowledge}{knowledge}:
\kl[kl-usage]{notion} points to its \intro[kl-usage]{definition}. 


% Include the content of the paper
\section{Introduction}
\label{sec:intro}

\AP Let us fix a finite set of indeterminates $\Indets$, a field $\Field$, and
let us consider the ring of polynomials $\Poly{\Indets}$. Important subsets of
a ring in commutative algebra include the \intro{(bilateral) ideals}, which are
subsets of the ring that are closed under addition and multiplication by any
element of the ring. Formally, an ideal $\idl$ of a commutative ring $R$ is a
subset of $R$ such that: for all $p,q \in \idl$ and $r \in R$, $r p + q \in
\idl$. A fundamental result from commutative algebra is \intro{Hilbert's Basis
Theorem}, which states that the ring of polynomials $\Poly{\Indets}$ is
\intro{Noetherian}, that is, every increasing chain of ideals in
$\Poly{\Indets}$ stabilises (see for instance \cite[Theorem 4]{CLO15}). An
equivalent formulation of this theorem is that every \kl{ideal} of polynomials
is \intro{finitely generated}, that is, there exists a finite set of
polynomials $p_1, \ldots, p_n$ such that $\idl = \idlGen{p_1, \ldots, p_n}$,
where $\intro*\idlGen{p_1, \ldots, p_n}$ is the smallest ideal containing $p_1,
\ldots, p_n$.

\AP Because every \kl{ideal} is \kl{finitely generated}, one can try to devise
algorithms to manipulate ideals of polynomials. One of the simplest algorithmic
question is to decide the \intro{ideal membership problem}, that is, given a
polynomial $p$ and an ideal $\idl$, decide whether $p \in \idl$. This problem
is known to be \EXPSPACE-complete \cite{MAME82}, and most of the algorithms
that are used in practice start by computing so-called \emph{Gröbner bases},
that represent the same ideal, but on which the ideal membership problem can be
more efficiently decided, by computing a linear number of polynomial
subtractions in the size of the Gröbner basis. Computing a Gröbner basis can
also be thought of a more algorithmically tractable presentation of \kl{ideals}
in general: not only \kl{membership} can be decided efficiently, but also
inclusions, computation of intersections and so on.

\AP Usually, the first algorithm that one encounters in the context of
\kl{Gröbner bases} is the one of \intro{Buchberger} \cite{BUCH76}, that
computes a Gröbner basis of an ideal $\idl$ from a finite set of its
gererators, and that is fundamentally based on three ingredients: a
\emph{multivariate} analogue of the Euclidean division algorithm,\footnote{
Recall that the ring of polynomials $\Poly{\Indets}$ is not a Euclidean domain
as soon as the number of indeterminates is greater than one.} a saturation
algorithm that bears some resemblance with the Knuth-Bendix completion
algorithm \cite{KNBE70}, and a termination proof based on a well-quasi-ordering
argument.

\AP There have been several attempts to adapt \kl{Hilbert's Basis Theorem} and
\kl{Buchberger's algorithm} to the case of an infinite number of
indeterminates. While the results fail in this case,\footnote{Gonsider the
ideal generated by all the indeterminates, it cannot be finitely generated} one
can still obtain interesting results when the set of indeterminates is equipped
with a group action. For instance, if all the indeterminates are considered to
be indistinguishable (that is, the group action consists of all permutations of
the indeterminates), then one recovers the \emph{up-to} version of
\kl{Hilbert's Basis Theorem} \cite{BRDR11,HIKRLE18}: every increasing chain of
\kl{ideals} (up-to the group action) stabilises. For instance, the ideal
generated by all the indeterminates is generated by a single one of them, up-to
the permutations of variables.

\AP Such approaches to better understand how these theorems can be adapted to
the presence of symmetries are interesting from a mathematical point of view,
but they also have a practical impact, for instance in the verification of
finite systems that manipulate abstract \emph{data} \cite{KAFR94}. 
In this more algorithmic context, the study of vector spaces up to a 
group action showed the pertinence of this approach \cite{BOZLMO21}.
Recently, the authors of \cite{GHOLAS24} proposed a rather general 
approach to the study of \kl{ideals} up to a group action, generalising
the results of \cite{BRDR11,HIKRLE18}.

\AP At the time of writing this document, the best approaches to the study of
\kl{ideals} up to a group actions deeply rely on the existence of a total
ordering on the indeterminates that is compatible with the group action. This
is for instance the case if variables are elements of $\Rat$, and the group is
composed of all monotone bijections of $\Rat$. However, this hypothesis is not
always satisfied, and in particular, it is not satisfied in the context of all
permutations (where one can use a trick to construct a total ordering on the
indeterminates by changing the group action). We believe that relying on the
existence of a total ordering on the indeterminates is an atavism, that
perdures because every proof of \kl{Hilbert's Basis Theorem} and
\kl{Buchberger's algorithm} that we know of starts by picking a total ordering
on the (finite set) of indeterminates, an action that is harmless in the finite
setting, and harmful in the infinite setting.

\paragraph{Contributions} \AP We provide a new self contained proof of
\kl{Hilbert's Basis Theorem} and an analogue of \kl{Buchberger's algorithm}
that do not rely on a total ordering of the indeterminates, in the case of a
finite number of indeterminates. Note that none of the results are new
\emph{per se}, but they focus on a point of view that was mostly ignored for
several reasons: first, in the finite case, the existence of a total ordering
is not harmful and simplifies the proofs; second, the results are deeply
non-constructive and rely multiple times on excluded middle and the axiom of
choice, and removing the need for a total ordering on the indeterminates is
somehow the least concern for constructivists.

\paragraph{Outline} \AP In
\cref{sec:hilbert}, we recall the
statement of \kl{Hilbert's Basis Theorem} and prove it without relying on a
total ordering of the indeterminates. In
\cref{alg:buchberger},
we present an alternative version of \kl{Buchberger's algorithm} that does not
rely on a total ordering of the indeterminates. Finally, in
\cref{sec:conclusion}, we discuss the main
limitations of our approach in the case of an infinite number of
indeterminates.


% LTeX: language=en
\section{Hilbert's Basis Theorem}
\label{sec:hilbert}

\AP Let $\Field$ be a field, and $\Indets$ a finite set of indeterminates. We
consider the ring of polynomials $\Poly{\Indets}$. We will denote by
$\Mon{\Indets}$ the set of \intro{monomials} over $\Field$ and $\Indets$, that are
products of indeterminates (without leading coefficient). Note that \kl{monomials}
are naturally quasi-ordered by divisibility, where $\mon \intro*\divleq \mon'$
if $\mon$ divides $\mon'$.

\AP A quasi-order $(X, \leq)$ is a \intro{well-quasi-order} if every infinite
sequence $x_0, x_1, \ldots$ of elements of $X$ contains two elements $x_i$ and
$x_j$ such that $i < j$ and $x_i \leq x_j$. In particular, this means that
there can be no infinite \intro{antichains} in $X$, that is, no infinite sets
of pairwise incomparable elements. Let us recall a consequence of Higman's
Lemma \cite{HIG52}: the set of monomials $\Mon{\Indets}$ is
\kl{well-quasi-ordered} by divisibility.

\AP To a polynomial $p \in \Poly{\Indets}$, we associate the set of maximal
monomials $\MaxMon(p)$, that is the set of monomials that are maximal for the
divisibility relation among the monomials of $p$. Remark that $\MaxMon \colon
\Poly{\Indets} \to \Pfin(\Mon{\Indets})$. Because $\Mon{\Indets}$ is
\kl{well-quasi-ordered} by divisibility, the set $\Pfin(\Mon{\Indets})$ is also
well-quasi-ordered by the \reintro[hoareDiv]{Hoare divisibility relation}: $S
\intro*\hoareDivleq S'$ if for all $M \in S$, there exists $M' \in S'$ such
that $M \divleq M'$ \cite{SCSC12}. We can use this to endow $\Poly{\Indets}$ with a
well-quasi-ordering $\intro*\pmonleq$: $p \pmonleq p'$ if $\MaxMon(p)
\hoareDivleq \MaxMon(p')$.

\AP One of the main remarks of this section is that one recovers a tight
correspondence between elements of subsets when comparing they \kl{upward}
(resp. \kl{downward}) closures. Let us recall that the
\intro{downward closure} of a set $S$ in a quasi-ordered set $(X, \leq)$
is the set $\intro*\dwset[\leq]{S} \defined
\setof{x \in X}{\exists s \in S, x \leq s}$, and the 
\intro{upward closure} of a set $S$ in a quasi-ordered set $(X, \leq)$
is the set $\intro*\upset[\leq]{S} \defined \setof{x \in X}{\exists s \in S, s \leq x}$.
When the ordering is clear from the context, we will omit the subscript $\leq$.

\begin{lemma}
    \label{lem:antichain-bijection}
    Let $(X,\leq)$ be a quasi-ordered set
    and let $S, S'$ be two finite subsets of $X$
    that are \kl{antichains} for $\leq$,
    and such that $\upset{S} = \upset{S'}$.
    Then, there exists a bijection $\sigma \colon S \tobij S'$
    such that $s \equiv \sigma(s)$ for all $s \in S$.

    The same result holds under the assumption that
    $\dwset{S} = \dwset{S'}$.
\end{lemma}
\begin{proof}
    Let $x \in S$, by assumption,
    there exists $y \in S'$
    such that $x \leq y$. But we also have
    an $x' \in S$ such that $y \leq x'$.
    Because $S$ is an \kl{antichain}, we have $x = x'$,
    and therefore $x \equiv y$.
    We can define $\sigma(x) \defined y$.
    Now, let us prove that $\sigma$ is injective.
    Assume that $\sigma(x) = \sigma(x')$.
    Then, $x \equiv \sigma(x) = \sigma(x') \equiv x'$,
  and because $S$ is an \kl{antichain}, we have $x = x'$.
    Let us now prove that $\sigma$ is surjective.
    This is because $S$ and $S'$ have the same size, that is
    the number of equivalence classes of minimal elements
    in $\upset{S} = \upset{S'}$.
    For the second part of the lemma, we apply
    the first part to the quasi-ordering $\geq$.
\end{proof}


\begin{lemma}
   \label{lem:polynomial-division}
   Let $p, q \in \Poly{\Indets}$ be two non-zero polynomials,
   such that $p \pmoneq q$.
    Then, there exists a constant $c \in \Field$ such that
    $(p - c \times q) \pmonlt q$.
\end{lemma}
\begin{proof}
    Let us first remark that $p \pmoneq q$ if and only if
     $\MaxMon(p) \hoareDivEq \MaxMon(q)$.
    Because $\MaxMon(p)$ and $\MaxMon(q)$ are antichains,
    one can apply \cref{lem:antichain-bijection} to find
    a bijection $\sigma \colon \MaxMon(p) \tobij \MaxMon(q)$
    such that $m \equiv \sigma(m)$ for all $m \in \MaxMon(p)$.
    Now, since the divisibility relation on monomials
    is antisymmetric (recall that monomials do not 
    have multiplicative constants coming from $\Field$), we have that $m = \sigma(m)$
    for all $m \in \MaxMon(p)$. 
    As a consequence, we can write $p = \sum_{m \in \MaxMon(p)} a_m \times m +
    r$ and $q = \sum_{m \in \MaxMon(q)} b_m \times m + r'$, where $a_m, b_m \in
    \Field$ and $R$ and $R'$ are the remaining terms in $p$ and $q$, i.e., are
    such that $r \pmonlt q$ and $r' \pmonlt q$.

    Now, one can select some $m_0 \in \MaxMon(p)$ (which is possible because
    $p$ and $q$ are non-zero polynomials), and write
    $q - c p = \sum_{m \in \MaxMon(q)} (b_m - c \times a_m) m + r' - c \times r$.
    If we select $c = b_{m_0}/ a_{m_0}$ (which is possible because 
    the numbers $a_{m_0}$ and $b_{m_0}$ are non-zero), then we have that 
    $m \notin \MaxMon(q - c \times p)$. However, by construction,
    every maximal monomial of $q - c \times p$ is in $\MaxMon(q)$.
    We conclude that $(q - c \times p) \pmonlt q$.
\end{proof}

\begin{theorem}[Hilbert's Basis Theorem]
    Let $\seqof{\idl_n}$ be an
    ascending chain of ideals in $\Poly{\Indets}$.
    Then, there exists an integer $N$ such that for all $n \geq \Nat$,
    $\idl_n = \idl_N$.
\end{theorem}
\begin{proof}
    Assume towards a contradiction that there is no such
    integer $N$, and without loss of generality, that
    for every $n \in \Nat$, $\idl_n \subsetneq \idl_{n+1}$.
    We can also assume that $\idl_0 \neq \{0\}$.

    Let us define $H_n \defined \upset[\pmonleq]{\idl_n}$ for all $n \in \Nat$.
    Because $\idl_n \subsetneq \idl_{n+1}$, it is clear that $H_n \subseteq
    H_{n+1}$. Because $\Poly{\Indets}$ is well-quasi-ordered by $\pmonleq$,
    every set $H_n$ can be written as $H_n \defined \upset[\pmonleq]{S_n}$ for
    some finite set $S_n \subseteq \Poly{\Indets}$ that is an antichain for
    $\pmonleq$. Let us remark that $0 \notin S_n$ for all $n \in \Nat$, and
    that $S_n$ in non-empty for all $n \in \Nat$.
    Without loss of generality, one can further assume that
    $S_n \subseteq \idl_n$ for all $n \in \Nat$.

    Since, $(\Poly{\Indets}, \pmonleq)$ is well-quasi-ordered,
    there exists an integer $N$ such that 
    $H_N = \bigcup_{n \in \Nat} H_n$.
    In particular, we have $\upset[\pmonleq]{S_N} = \upset[\pmonleq]{S_{N+1}}$.
    Applying \cref{lem:antichain-bijection}, we find a bijection
    $\sigma \colon S_N \tobij S_{N+1}$ such that $p \pmoneq \sigma(p)$
    for all $p \in S_N$.
    Using the fact that $S_N$ is non-empty and that $0 \notin S_N$,
    we can find a non-zero polynomial $p \in S_N$,
    and its corresponding $q \defined \sigma(p) \in S_{N+1}$.

    By \cref{lem:polynomial-division}, we can find a constant $c \in \Field$
    such that $(p - c \times q) \pmonlt q$. Now, because 
    $p \in S_N \subseteq \idl_N \subseteq \idl_{N+1}$ 
    and $q \in S_{N+1} \subseteq \idl_{N+1}$,
    we have that $(p - c \times q) \in \idl_{N+1}$.
    This contradicts the minimality of $q$ in $H_{N+1}$.
\end{proof}

% LTeX: language=en
\section{Buchberger's Algorithm}
\label{sec:buchberger}

In this section, we present a version of Buchberger's algorithm that does not
require a total ordering on the indeterminates. This will generalise the proof
of Hilbert's Basis Theorem done in \cref{sec:hilbert}. Let us first recall the
intuition behind Buchberger's algorithm, and the actual problem it is solving.

Given a finite set $H$ of polynomials, and a polynomial $p$, we want to check
whether $p$ is in the ideal generated by $H$. More generally, we would like to
have a finite representation of the ideal generated by $H$ over which it is
easy to compute such membership queries, unions, intersections, etc. Note that
the membership problem is generalises reachability and is therefore EXPSPACE.
On more efficient representations, called \kl{Gröbner bases}, the same
problem is in lintime. In this setting, Buchberger's algorithm is a
precomputation procedure.

\AP Let us first provide a naïve but incomplete algorithm for membership
testing: to any set $H$ of polynomials, one can associate a rewriting system
$\intro*\premulteucl{H}$ that is representing \emph{multiviriate Euclidian
division}. A transition of the form $p \premulteucl{H} r$ means that there
exists a \kl{monomial} $\mon$ and a constant $a$ such that $p = a \mon q + r$
for some $q \in H$. It is easy to see that $p \in \idlGen{H}$ if and only if $p
\premulteucl{H}^* 0$. 

\AP
However, due to the fact that this rewriting system is reversible, it is not
terminating and does not yield a decision procedure for the $\idlGen{H}$
membership. One way to ensure termination is to select a \kl{well-founded}
\kl{total ordering} $\monord$ on monomials that is \kl{compatible with the multiplicative
structure}, and restrict the rewriting system to the transitions that strictly
decrease in this ordering. This defines
a sub-transition system $\multeucl{H}{\monord}$ that is no longer reversible
by construction, and is furthermore \intro{terminating}: there are no
infinite sequences of reductions in the system.

\AP A (pre)order $\leq$ is a \intro{monomial ordering}
if for all monomials $\mon_1, \mon_2, \mon_3$, such that $\mon_1 \leq \mon_2$,
$\mon_1 \mon_3 \leq \mon_2 \mon_3$. The existence of \kl{well-founded}
\kl(order){total} \kl{monomial orderings} follows from a relatively
straightforward result in order theory: for every $k \in \Nat$, $\Nat^k$ is a
\kl{well-quasi-order} when endowed with the componentwise ordering.


\begin{lemma}[Folklore]
    \label{lem:monomial-order-exists}
    For every finite set $\Indets$ of indeterminates,
    there exists a \kl{well-founded}
    \kl(order){total} \kl{monomial orderings}. Furthermore,
    it is computable.
\end{lemma}
\begin{proof}
    Select any total ordering $<$ on the finite set of
    indeterminates $\Indets$. We define our total ordering
    as the lexicographic ordering obtained when considering 
    monomials as sequences of numbers.
    It is clearly a \kl(order){total}
    \kl{monomial ordering}.
    Because of Dickson's Lemma \cite{SCSC12},
    the set of monomials endowed with the \kl{divisibility preordering}
    is a \kl{well-quasi-order}.
    Since our ordering is a linearisation of the 
    \kl{divisibility preordering} (that is, if
    $\mon_1 \divleq \mon_2$, then $\mon_1 \leq \mon_2$),
    it must therefore be \kl{well-founded}
    \cite{SCSC12}.
\end{proof}

\AP In order to prove that a given rewrite system $\multeucl{H}{\leq}$ is
\kl(eucl){complete}, it is interesting to notice the connection between this
property and a classical notion in rewriting theory, namely \kl{confluence}. A
rewriting system $\to$ is \intro{confluent} whenever, for all triples $x,y,z$
such that $x \to y$ and $x \to z$, there exists a $t$ such that $y \to^* t$ and
$z \to^* t$. Note that, as for the \kl(eucl){completeness}, there is no reason
for a relation $\multeucl{H}{\leq}$ to be \kl{confluent} (see
\cref{fig:buchberger-non-confluent}).

\begin{figure}
    \centering
    \begin{tikzpicture}
        \node (p) {$xyz$};
        \node[below=of p] (r1) {$y$};
        \node[right=of p] (r2) {$z$};

        \path[-stealth]
        (p) edge node[font=\tiny,left]{$(H,\leq)$} (r1)
        (p) edge node[font=\tiny, midway, above]{$(H,\leq)$} (r2);
    \end{tikzpicture}
    \caption{
        An example of non confluence
        for the rewriting system
        $\multeucl{H}{\leq}$, where $H = \set{ xy, xz }$,
        for any choice of \kl{monoidal ordering}
        $\leq$.
    }
    \label{fig:buchberger-non-confluent}
\end{figure}

One particularity of the directed rewrite systems is that when a polynomial $p$
rewrites to zero, not only do we conclude that $p \in \idlGen{H}$, but also
that $p \in \idlGen{H_{\leq p}}$, where $H_{\leq p}$ is the set of elements in
$H$ that are less or equal to $p$.

\begin{lemma}
    \label{lem:confl-impl-complete}
    Let $H$ be a finite set of polynomials
    and $\leq$ be a \kl{well-founded} \kl(order){total}
    \kl{monomial ordering}.
    If the rewriting system $\multeucl{H}{\leq}$ is \kl{confluent}
    then it is \kl(eucl){complete}.
\end{lemma}
\begin{proof}
    Assume towards a contradiction that
    there is a polynomial $p \in \idlGen{H}$
    that does not rewrite to $0$.
    Because $\leq$ is a \kl{total} and \kl{well-founded},
    one can assume $p$ to be minimal for this property.

    Since $p \in \idlGen{H}$,
    there exists a sequence $\seqof{a_i}[1 \leq i \leq n]$
    of non-zero coefficients, 
    a sequence $\seqof{\mon_i}[1 \leq i \leq n]$
    of monomials,
    and a sequence 
    $\seqof{h_i}[1 \leq i \leq n]$ of elements of $H$
    such that:
    \begin{equation}
        p = \sum_{1 \leq i \leq n} a_i \mon_i h_i \quad .
    \end{equation}

    Let $\mon[K]$ be maximal among the monomials of $p$,
    and $\mon[L]$ be maximal among the monomials
    of the polynomials $\mon_i h_i$ for $1 \leq i \leq n$.
    Without loss of generality, one can also assume that 
    $\mon[L]$ is minimal among all representations of $p$.
    There are two cases:

    \begin{description}
        \item[The two monomials are equal.]
           In this case, $\mon[K]$ is the maximal monomial of
           \emph{some} $\mon_i h_i$.
           Then, let us write $\beta$ for the coefficient of
           $\mon[K]$ in $p$,
           and
           $r \defined p - \beta \mon_i h_i$.
           It is clear that 
           $r < p$, and that $p \premulteucl{H} r$,
           hence that 
           $p \multeucl{H}{\leq} r$.

           We conclude that $r \in \idlGen{H}$, 
           and by minimality of $p$, $r \multeucl{H}{\leq}^* 0$.
           This proves that 
           $p \multeucl{H}{\leq}^* 0$ 
           which is absurd.

        \item[The two monomials are not equal.]
            
            Let us write $J$ the set of indices $1 \leq j \leq n$
            such that $\mon[L]$ is the maximal monomial
            of $\mon_j h_j$. Let us also assume that the
            (non-zero) coefficient of the monomial $\mon[L]$
            in $\mon_j h_j$ is $1$ (that can be done by changing
            the $a_i$'s).
            Because $\mon[L] \neq \mon[K]$,
            we conclude that
            \begin{equation}
                \sum_{j \in J}
                a_j = 0 \quad .
            \end{equation}
            Because the $a_j$'s are non-zero and
            $J$ is non-empty, we conclude that $J$ contains at least two 
            elements. Let us call
            $J^1$ the set $J$ where one element has been 
            removed. To avoid nested indices, we call this
            element $\star$.

            Let us rewrite the polynomial $p$
            as follows, letting $I = \set{ 1, \dots, n}$:
            \begin{equation}
                p = \sum_{j \in J^1} a_j (\mon_j h_j - \mon_\star h_\star)
                  + 
                    \sum_{i \in I \setminus J} a_i \mon_i h_i
                  \quad .
            \end{equation}

            Now, because the rewriting system is confluent,
            and because
            $\mon[L]$
            can be rewritten into 
            $\mon[L] - \mon_j h_j$
            for all $j \in J$,
            there exists a polynomial $t$ such that
            $\mon[L] - t = 
            \mon_j h_j +  r_j$ where $r_j$ is obtained as a combination of elements 
                    in $H$ having maximal monomials strictly below $\mon[L]$.
            As a consequence,
            \begin{equation*}
                \mon_j h_j - 
                \mon_\star h_\star
                =
                r_\star
                -
                r_j
                \quad .
            \end{equation*}

            We have obtained a new way to write $p$
            using elements of $H$ having a strictly
            smaller maximal monomial. This contradicts
            the minimality of $\mon[L]$.
    \end{description}
\end{proof}

Let us prove the converse direction, i.e. that \kl(eucl){completeness}
implies \kl{confluence}. This will be crucial in understanding the
Buchberger algorithm.

\begin{lemma}
    \label{lem:confl-iff-corr}
    Let $H$ be a finite set of polynomials and 
    $\leq$ be a \kl{well-founded}
    \kl(order){total}
    \kl{monomial ordering}. Then, the following are equivalent:
    \begin{enumerate}
        \item \label{item:complete}
            The rewriting system $\multeucl{H}{\leq}$ is
            \kl(eucl){complete}.
        \item \label{item:confl}
            The rewriting system $\multeucl{H}{\leq}$ is
            \kl{confluent}.
        \item \label{item:wconfl}
            The rewriting system $\multeucl{H}{\leq}$ is
            \kl{confluent} when starting from monomials.
    \end{enumerate}
\end{lemma}
\begin{proof}
    It is clear that \cref{item:confl}
    implies \cref{item:wconfl}. Furthermore,
    \cref{lem:confl-impl-complete} actually only used the
    confluence when starting from monomials,
    effectively proving \cref{item:wconfl}
    implies \cref{item:complete}.

\end{proof}

A consequence of \cref{lem:confl-impl-complete} is that the membership problem
becomes very easy to solve: keep reducing until no more steps can be taken, and
check if the normal form is $0$.

The classical method to ensure confluence is to 
understand the \intro{critical pairs} of the rewriting
system, which essentially amounts to

\subsection{Orderless Buchberger}

The only idea here is to not use a \kl{monoidal ordering} but
to use a \kl{divisibility partial ordering} instead. Note that because
the latter in not total, one has to be careful about 
the precise statements. However, we will show that if one
orders polynomials according to their set of maximal monomials
(with the powerset ordering), then one can redo the confluence analysis
to obtain a complete rewriting system.

\subsection{Atomic Buchberger?}

\section{Concluding Remarks}
\label{sec:conclusion}


We wonder whether some of the results in this paper could be extended to
infinite sets of variables, such that the monomials are still
well-quasi-ordered by divisibility. This is a natural question that arises in
the context of sets with atoms, where the set of variables is infinite but are
considered up to a group of automorphisms.

\begin{itemize}
  \item List uses 
\end{itemize}



% Include the bibliography
\bibliographystyle{plainurl}
\bibliography{papers.bib}

% If there are any appendices, we include them here.
\appendix
\input{src/appendix.tex}


\end{document}
