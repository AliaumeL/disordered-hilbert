% Little math macros
\NewDocumentCommand{\set}{ m }{\{ #1 \}}
\NewDocumentCommand{\setof}{ m m }{\{ #1 \mid #2 \}}
\NewDocumentCommand{\card}{ m }{\left| #1 \right|}
\NewDocumentCommand{\seqof}{ m O{n \in \Nat} }{\left( #1 \right)_{#2}}

\NewDocumentCommand{\defined}{ }{\triangleq}
\newcommand{\defiff}{\overset{\mathrm{def}}{\iff}}
\newcommand{\defeq}{\overset{\mathrm{def}}{=}}

\newcommand{\subfin}{\subset_{\text{fin}}}

\newcommand{\Pfin}{\mathcal{P}_{\text{fin}}}
\newcommand{\Pset}{\mathcal{P}}

% functions of all sorts (injective, partial, surjective)
\newcommand{\topartial}{\rightharpoonup}
\newcommand{\toinj}{\hookrightarrow}
\newcommand{\tosurj}{\twoheadrightarrow}
\newcommand{\tobij}{\stackrel{\simeq}{\longrightarrow}}

% classical sets
\newcommand{\Nat}{\mathbb{N}}

% Automate the creation of new orderings
% based on a given symbol.
% For instance,
% \NewDocumentOrdering{\pref}{\preceq}{\prec}
% will create the following commands:
% \prefleq and \preflt
% that will respectively expand to
% \mathrel{\kl[\pref]{\preceq}} and \mathrel{\kl[\pref]{\prec}}
\NewDocumentCommand{\NewDocumentOrdering}{ m m m }{
    \expandafter\newcommand\csname #1leq\endcsname{
        \mathrel{\kl[#1]{#2}}
    }
    \expandafter\newcommand\csname #1lt\endcsname{
        \mathrel{\kl[#1]{#3}}
    }
    \knowledge{#1}{notion}
}

% Order macros
\NewDocumentCommand{\upset}{ O{} m }{{\uparrow_{#1} #2}}
\NewDocumentCommand{\dwset}{ O{} m }{{\downarrow_{#1} #2}}


% Number theory
\NewDocumentCommand{\factorial}{ O{} m }{
    \if\relax\detokenize{#1}\relax
        #2!
    \else
        (#2)!
    \fi
}

% orders 
\NewDocumentOrdering{div}{\sqsubseteq^{\mathrm{div}}}{\sqsubset^{\mathrm{div}}}

\NewDocumentOrdering{pmon}{\preceq}{\prec}
\NewDocumentCommand{\pmoneq}{}{\mathrel{\kl[pmon]{\simeq}}}

\NewDocumentOrdering{hoareDiv}{\sqsubseteq^{\mathrm{div}}_{\flat}}{\sqsubset^{\mathrm{div}}_{\flat}}
\NewDocumentCommand{\hoareDivEq}{}{ \mathrel{\kl[\hoareDiv]{\equiv^{\mathrm{div}}_{\flat}}} }

\newcommand{\Field}{\mathbb{K}}
\newcommand{\Indets}{\mathcal{X}}
\NewDocumentCommand{\Poly}{ O{\Field} m }{#1[{#2}]}
\NewDocumentCommand{\Mon}{ O{\Field} m }{\mathsf{Mon}_{#1}({#2})}

\NewDocumentCommand{\mon}{ O{M} }{#1}
\NewDocumentCommand{\pol}{ O{p} }{#1}


\newcommand{\idlGen}[1]{\langle #1\rangle}
\newcommand{\idl}{\mathcal{I}}

\newcommand{\MaxMon}{\mathsf{MaxMon}}


